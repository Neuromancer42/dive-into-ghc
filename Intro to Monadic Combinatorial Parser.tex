\documentclass{article}
\usepackage{amsmath}
\usepackage{ctex}
\usepackage{listings}
\usepackage{cite}

\title{Monadic Parser Combinators介绍}
\author{陈逸凡 ( Neuromancer )}
\date{19th Mar, 2017}

\lstset{language=haskell,frame=single}

\begin{document}
\maketitle

\section{Intro}
  \paragraph{提示}
   本篇假设读者对基本的 Haskell 语法有所了解,主要包括:类型匹配,参数化类型,lambda 表达式,类型构造器,类型类等。不需要读者熟悉 Monad 类型。  
 \subsection{何为 Combinator}
  \paragraph{}
   在 Parser 构造过程中,将目标 Parser 分解为多个小的 Parser 再组合起来是一种常用的构造方式。将大的问题分解为多个小目标,再把对这些小目标构造出的各种简单 Parser 像乐高积木一样组合,便能最终得到一个复杂而精美的 Parser,这便是“分解——组合”思想的精妙之处。
  \paragraph{}
   Combinators 将 Parser 的“组合”抽象成一个 Domain Specific Language (DSL) 。用统一的语言来表示这样的组合过程,并提供了一些最基本的 Parser ,这就使得构造 Parser 变成了一个非常简单而易读( readable) 的过程。而且各种 Parser 都可复用同样的函数,也很符合代码复用的目标。
 \subsection{何为 Monadic}
  \paragraph{}
   Monad是函数式编程中常用的一类抽象数据类型/代数概念,这里暂且不提其抽象性质,只谈一谈其在Parsing过程中的用途。
  \paragraph{}
   假设已经有了一个简单的 Parser 记为\textbf{p},对于给定输入(这里类型不妨为 String ,后同),\textbf{p} 要么将其解析出一个类型为 a 的结果(和剩余的未解析的输入),要么给出一个错误信息。我们可以将 \textbf{p} 的类型其表示为以下用 Haskell 代码表示的形式。
   \begin{lstlisting}
   data Parser a = 
     P { runParser : Input -> ParseResult a }
   data ParseResult a = Err ParseErr | Res Input a 
   \end{lstlisting}
  \paragraph{}
   使用 monad (或者在 Haskell 语境下,将 Parser 定义为 Monad 类型类实例),可以达到这样的效果:
   \begin{enumerate}
	\item 如果成功解析,可以把解析出的数据传递下去并被后续的解析过程利用,利用高阶函数,可以充分重用解析结果类型 a 可用的各类函数而不必一一手动提升
	\item 如果解析失败,可以把错误信息传递下去而并不需要特殊处理,并在最后用统一的错误处理过程来解决出错信息,这样相比直接 error 跳出或显式地传递错误信息对重构代码要友好得多
	\item 还可以为解析结果附加更多的有用信息,具体做法可以是给 Parser a 类型再套一个 Logger 类型类。值得一提的是,这么做的时候,只需要修改 Parser 的定义和添加一个统一的 Logger 信息处理过程,而几乎不必修改其他部分代码。不过这里还是用当前不带 Logger 的 Parser 类型,目的是只关注本篇主题。
   \end{enumerate}	
  \paragraph{}
   Monad 到底能带来多大的益处,可以通过后面的讲解来详细展示。

\section{基本组合子的构造}
  \paragraph{}
   就像乐高积木一样,Parser Combinator 也会提供一些最简单的砖块以供使用者来构造更复杂的结构。
 \subsection{固定读取一个字符的 Parser}
  \paragraph{}
   就跟字面意思一样,这个 Parser 固定从输入流头部取出一个字符作为结果;当输入为空时,就返回错误信息。
  \begin{lstlisting}
  anyChar :: Parser Char
  anyChar = P $ \inp ->
    case inp of
      [] -> Err UnexpectedEof
      x:xs -> Res xs x
  \end{lstlisting}
 \subsection{永远返回错误信息的 Parser}
  \paragraph{}
   由于错误信息有多种,所以需要构造多个对应的 Parser,这里举两个例子,有需要的时候可以自己添加更多错误信息到 ParseError 类型中,并构造出对应的组合子。
  \begin{lstlisting}
  unexpectedChar :: Char -> Parser a
  unexpectedChar c = P $ \_ -> Err (UnexpectedChar c)
  
  failed :: Parser a
  failed = P $ \_ -> Err Failed
  \end{lstlisting}
 \subsection{不解析但返回一个值}
  \paragraph{}
   这个 Parser 实际上并不解析任何东西,构造这个组合子的目的在于将把基本类型的计算过程引入到 Parser 里。具体用法后面会再出现。
  \begin{lstlisting}
  pureParser :: a -> Parser a
  pureParser x = P $ \inp -> Res inp x
  \end{lstlisting}
  \paragraph{}
  这就是全部所需的基本 Parser 。就如同所看到的那样,非常简洁。可以类比为这就是一个代数系统的生成元,配合给出的运算子,就可以组合出全部可能的 Parser 。
 \section{定义基本运算}
   \paragraph{}
    以上组合子,实际发生解析的只有第一个 anyChar 解析器,要怎样从找一个简陋至极的解析器出发构造更多的解析器呢?这就需要一些运算来调整 anyChar 了。
  \subsection{将基本函数映射到 Parser 上}
   \paragraph{}
    需要一个“提升”操作,使得原本作用在 a 类型上的函数应用到 Parser a 上。
   \begin{lstlisting}
   mapParser :: (a -> b) -> Parser a -> Parser b
   mapParser f p = P $ \inp ->
     case runParser p inp of
       Res inp' x -> Res inp' (f x)
       Err err -> Err err
   \end{lstlisting} 
   \paragraph{}
    这里,我们用一个基本函数和一个 Parser 构造出一个产生不同结果的 Parser 。如果原 Parser 会解析错误,那么新 Parser 也返回同样的解析错误;如果原 Parser 解析出一个结果,那么新 Parser 就会给出函数作用在结果后的返回值。
  \subsection{将解析结果提供给之后的解析过程}
   \paragraph{}
    如果我们需要把解析出来的结果提供给后续的解析过程使用,比如构造带有复杂数据结构的结果,又或者是上下文敏感文法(context sensitive)需要,就需要一个操作来“取出”解析结果。
   \begin{lstlisting}
   bindParser :: 
     (a -> Parser b) -> Parser a -> Parser b
   bindParser bf pa = P $ \inp ->
     case runParser pa inp of
       Res inp' x -> runParser (bf x) inp'
       Err err -> Err err
   \end{lstlisting} 
   \paragraph{}
    这个函数的含义是,给定一个由一个参数决定的 Parser ,和一个 Parser ,将后者的解析结果作为参数提供给前者,得到一个新的 Parser ,——这与之前的“取出结果提供给后续过程”是等价的。同之前的 mapParser 一样,如果第一个 Parser 就解析错误,那后续过程也不会继续,而是直接返回解析错误。
   \paragraph{提示}
    为方便书写,后面使用交换两个参数位置的 flbindParser 。
   \begin{lstlisting}
   flbindParser :: 
     Parser a -> (a -> Parser b) -> Parser b
   flbindParser = flip bindParser
   \end{lstlisting}
   
  \subsection{定义为类型类实例}
   \paragraph{}
    事实上,已经定义的这些运算已经足以定义 Parser 为类型类实例。这样做的目的在于,一来只需要定义基本函数就可以充分利用一系列 Haskell 已定义的衍生函数;二来可以用统一的 API 来提供给一个,比如说,对 Monad 已经有所了解的使用者,让TA几乎不看文档就可以直接利用 Monad 类型类的操作符来操作 Parser 。
   \begin{lstlisting}
   instance Functor Parser where
     fmap :: (a -> b) -> Parser a -> Parser b
     fmap = mapParser
     
   instance Applicative Parser where
     pure :: a -> Parser a
     pure = pureParser  
     (<*>) :: Parser (a -> b) -> Parser a -> Parser b
     pf <*> pa = flbindParser pf $ \f ->
                 flbindParser pa $ \x ->
                 pureParser $ f x
   
   instance Monad Parser where
     (=<<) :: (a -> Parser b) -> Parser a -> Parser b
     (=<<) = bindParser 
   \end{lstlisting}
   \paragraph{}
    感兴趣的读者可以自行验证在这里 Functor law ,Applicative law 和 Monad law 都得到了满足。这也是我们标题中第一个单词的来源。由类型类导出的函数会在后面使用时再做出对应解释。
   \subsection{串联解析器}
   \paragraph{}
   把两个解析器串联起来,前一个解析完了让第二个继续解析。这样的操作都已经由 Applicative 类型类提供了对应的操作符,分别是 <*> 将前一个解析结果应用到后一个解析结果上作为返回值,*> 只返回后一个 Parser 的解析结果,<* 只返回前一个 Parser 的解析结果。
   \subsection{并联解析器} 
   \paragraph{}
   如果第一个 Parser 解析失败,那么就尝试用第二个 Parser 解析。
   \begin{lstlisting}
   (<|>) :: Parser a -> Parser a -> Parser a
   p <|> q = P $ \inp ->
   case runParser p inp of
   Err err -> runParser q inp
   res -> res
   \end{lstlisting}
   \paragraph{}
   有了并联操作之后,最有趣的一点在于我们可以借此实现解析数量不定的元素的 Parser ,实现正则表达式中符号 * / +,或 EBNF 表达式中符号 {} 的功能。
   \begin{lstlisting}
   manyP :: Parser a -> Parser [a]
   manyP pa = someP pa <|> pure []
   
   someP :: Parser a -> Parser [a]
   someP pa = pa >>= \x ->
   manyP pa >>= \xs ->
   pure $ x:xs
   \end{lstlisting}
   \paragraph{}
   这是一个间接递归调用的例子,many返回包含0或多个元素的列表,some返回包含至少一个元素的列表。
   \paragraph{}
   在定义了并联操作和 some ,many 操作后,实际上我们已经使得 Parser 满足了一个称为 MonadPlus 的类型类 / 代数结构的要求——也就是目前所看到的这些要求。 
 \section{更多操作}
   \paragraph{}
    通过之前定义的这些基本操作,我们可以定义出更多更复杂的操作。
  \subsection{只接受限定字符}
   \paragraph{}
    对 anyChar 解析出的字符进行筛选。
   \begin{lstlisting}
   satisfyP :: (Char -> Bool) -> Parser Char
   satisfyP p = anyChar >>= \c ->
     if p c 
       then pure c
       else unexpectedChar c
   \end{lstlisting}
   \paragraph{注}
    '>>=' 操作符就是 flbindParser 的中缀版本。
   \paragraph{}
    类似的,可以把这里的 Char 类型替换为更一般的类型变量,不过就不能使用 unexpectedChar 作为解析错误信息了。
   \paragraph{}
    通过替换这里的判定函数 p ,也可以构造许多特定的筛选函数。如:
   \begin{lstlisting}
   isP :: Char -> Parser Char
   isP c = satisfyP (== c)
   
   digitP :: Parser Char
   digitP :: satisfy isDigit
   \end{lstlisting}
  
  \subsection{按顺序调用一列解析器}
   \paragraph{}
    把同一类型的 Parser 放在一个列表中,按序调用后返回一个解析结果列表,这样可以简化不少重复劳动。
   \begin{lstlisting}
   seqParser :: [Parser a] -> Parser [a]
   seqParser [] = pure []
   seqParser (p:ps) = p >>= \x ->
                      seqParser ps >>= \xs ->
                      pure $ x:xs
   \end{lstlisting}
   \paragraph{}
    这样,我们可以直接由一个字符串生成一个解析指定字符串的 Parser:
   \begin{lstlisting}
   stringP :: String -> Parser String
   stringP s = seqParser $ map isP s
   \end{lstlisting}
  \subsection{解析被特定符号包裹的数据}
   \paragraph{}
    这样的 Parser 的应用场景可以是比如读取一个 HTML tag 。
   \begin{lstlisting}
   betweenP :: 
     Parser b -> Parser c -> Parser a -> Parser a
   betweenP pl pr pm = pl *> pm <* pr
   \end{lstlisting}
   \paragraph{}
    上面这段代码含义就是左中右顺序解析,并忽略掉两侧解析特殊符号的结果,只返回中间的解析结果。
  \subsection{解析被特定符号分割的数据}
   \paragraph{}
    使用在解析比如一列逗号分割的字符串中,如 CSV 格式。
   \begin{lstlisting}
   sepByP :: Parser b -> Parser a -> Parser [a]
   sepByP pb pa = ( pa >>= h ->
                  many (pb *> pa) >>= t ->
                  pure $ h : t ) <|> pure []
   \end{lstlisting}  
   \paragraph{}
    上面这段代码,首先读取 1 个元素,再识别 0 个或多个前面包含分隔符的元素,并把所有解析出元素串起来。如果一个元素都没有识别到,就返回空列表。
  \subsection{更多更复杂的 Parser }
   可以看到,在定义基本操作之后,定义其他 Parser 就不再涉及具体的 Parser 类型类的定义。作为 Parser Combinator 的使用者,并不需要知道 Parser a 的具体结构,只需要知道 Parser Combinator 提供的基本组合子和操作符,就可以搭建出一个解析出特定结构的 Parser ,这也就是为什么这些 Combinators 可以被称之为 Embedded Domain Specific \textit{Language} 而不仅仅是一个 Library 。
 \section{一个 JSON Parser 的例子}
   \paragraph{}
    下面,就用以 JSON 格式为例,看我们如何方便地构造将字符串解析成用 Haskell 抽象数据结构表示的 JSON 类型的。\footnote{基于 data61/fp-course 的代码。https://github.com/data61/fp-course}
   \lstinputlisting{JSONParser.hs}
   \paragraph{}
    这个 Parser 的构造仅需要数十行,其中甚至有超过三分之一是对特殊字符 escape 的处理(注,这里略去了一些小的操作函数的定义,但从其名字中也很容易看出来是怎么用组合子构造的。此外还略去了对特殊字符的转换函数)。相比之下,一个用 C 语言手写的递归下降解释器就比这复杂得多了,可读性和代码复用性也要差上不少。
   \paragraph{}
    值得一提的是这里对 JSON Number Parser 的构造。我们并没有用已有的组合子,而是自己重新构造了一个 Parser ,因为可以直接利用已有的 read 函数。于是这又体现出 \textit{Embedded} DSL 的另一个特性,即可以与宿主语言无缝结合。
 \section{更复杂的问题}
   \paragraph{}
    之前的 JSON Parser 是非常简单的,因为这是一个 LL(1) 文法,简单的递归下降解释器就能完成解析。对于更复杂的问题,Parser Combinators 同样可以解决。
  \subsection{上下文相关性}
   \paragraph{}
    如前所述,利用 Monad 的 bind 功能可以将 Parser 的解析结果取出提供给后续的解析过程,这也就解决了上下文相关的问题。当然,这也是递归下降解释器都能做到的事情
  \subsection{任意向前看文法}
   \paragraph{}
    之前定义的 <|> 函数实际上已经隐含了无限向前看的能力。第一个 Parser 可以尝试解析前方任意长字符串,失败了就交给第二个 Parser 重新。
  \subsection{左递归处理}
   \paragraph{}
    实际上,Parser Combinator 如何处理左递归 left-recursion 直到 2008 年才解决,在 Parser Combinator 概念第一次提出的 19 年后。\footnote{Frost, Richard A.; Hafiz, Rahmatullah; Callaghan, Paul (2008). "Parser Combinators for Ambiguous Left-Recursive Grammars".}
   \paragraph{}
    对于左递归的处理,可以参考 parsec 库\footnote{https://github.com/aslatter/parsec}的做法,这里直接引用其中的代码:
   \begin{lstlisting}
   
   chainl1 :: (Stream s m t) => ParsecT s u m a 
     -> ParsecT s u m (a -> a -> a) 
     -> ParsecT s u m a
   chainl1 p op = do{ x <- p; rest x }
                where
                  rest x    = do{ f <- op
                                ; y <- p
                                ; rest (f x y)
                                }
                              <|> return x

   chainr1 :: (Stream s m t) => ParsecT s u m a 
     -> ParsecT s u m (a -> a -> a) 
     -> ParsecT s u m a
   chainr1 p op = scan
                where
                  scan = do{ x <- p; rest x }
                  rest x    = do{ f <- op
                                ; y <- scan
                                ; return (f x y)
                                }
                              <|> return x
   \end{lstlisting}
  \subsection{二义性文法}
   \paragraph{}
    对于二义性,最直接的方法就是将所有可能结果存放在一个列表中,将 Parser 定义改为 P (Input -> [ParseResult]) (这样的改动并不会带来大范围重构),用一个类似于 <|> 操作符的函数将所有可能结果合并起来。但这样的代价就是可能会导致复杂度急剧上升,用 Memoization 可以使得复杂度降至多项式时间\footnote{Frost, Richard A.; Szydlowski, Barbara (1996). "Memoizing Purely Functional Top-Down Backtracking Language Processors"}。
    
 \section{总结}
  \paragraph{}
   Parser Combinators 是一项非常简洁的自顶向下 Top-Down 语法分析器构造技术,利用这项技术可以极大减小编写语法分析器时的痛苦。利用一些高效的基础数据结构,可以使得 Parser Combinators 有着可与自底向上解析器一比的效率。
 \nocite{*}
 \bibliography{monParseRef}
 \bibliographystyle{plain}    
\end{document}