\documentclass{article}
\usepackage{ctex}
\usepackage{amsmath}
\usepackage{listings}
\usepackage{biblatex}

\author{陈逸凡,吴昊泽}
\title{GHC 源码剖析}

\addbibresource{GHCRef.bib}
\lstset{language=haskell,frame=single}

\begin{document}
	\maketitle
	\section{GHC 整体概况}
	\section{语法解析}
	\paragraph{简述}
	GHC 在前端使用 Alex\cite{alex} 和 Happy\cite{happy} 这两个在 Haskell 开发中常用的工具来生成词法解析器和语法解析器。在这一模块的源文件\cite{ghcparser}中,除了词法分析器 ( Lexer.x ) 和语法分析器 ( Parser.y ) 的生成文件之外,还包括了辅助的语法定义文件 ( RdrHsSyn.hs ),以及处理 C-FFI ( C Foreign function interface ) 和 Haddock 文档生成所需的辅助函数等。
\end{document}