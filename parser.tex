\documentclass{article}
\usepackage{ctex}
\usepackage{minted}
\usepackage{amsmath}
\usepackage{biblatex}

\author{陈逸凡,吴昊泽}
\title{GHC 源码剖析:语法分析}

\addbibresource{GHCRef.bib}

\begin{document}
	\maketitle
	
	\section{概要}
	\paragraph{}
	GHC 在前端使用 Alex\cite{alex} 和 Happy\cite{happy} 这两个在 Haskell 开发中常用的工具来生成词法分析器和语法分析器。在这一模块的源文件\cite{ghcparser}中,除了词法分析器 ( Lexer.x ) 和语法分析器 ( Parser.y ) 的生成文件之外,还包括了辅助的语法定义文件 ( RdrHsSyn.hs ),以及处理 C-FFI ( C Foreign function interface ) 和 Haddock 文档生成所需的辅助函数等。接下来主要介绍 alex 与 happy 这两个工具和 GHC 在语法解析过程中一些比较重要的额外处理。
	\section{Alex}
	\subsection{功能和结构}
	\paragraph{}
	Alex\cite{alex} 是一个主要用 Haskell 开发的词法解析器生成器。功能与 C 语言开发常用的 Lex 或 Flex 类似,将用户定义的描述文件转换为包含生成的扫描器函数的 Haskell 源码。一个描述文件主要包括前缀代码,wrapper 声明,宏定义,规则和后缀代码这些部分。前缀代码和后缀代码为 Haskell 代码,主要功能有声明导入模块,定义导出模块,定义全局类型等。需要注意的是在 Alex 中,用户需要自己声明和定义 token 的类型和值,一般放置在后缀代码这一板块中。
	\subsection{规则}
	\paragraph{}
	由于 Haskell 是静态类型的纯函数式语言,所以相比 Lex 中的规则,Alex 描述文件的规则中模式所对应的代码有更严格的要求,每条规则对应的动作代码,应当是符合约定类型的函数。
	\paragraph{}
	对于要求动作函数满足约定类型这一较强的限制,Alex 提供了 wrapper 声明的语法,wrapper 声明的作用在于,约定了每个动作的类型,从而使得用户可以使用预定义的高阶 API,免去自定义的麻烦。不同的 wrapper 对应不同样式的函数类型,如 "basic" wrapper 约定每条规则的输入类型为 String 输出类型为自定义的 token 类型;"posn" wrapper 则在输入中提供了匹配串的位置信息;"monad" wrapper 和 "monadUserState" wrapper 则提供了传递全局状态的 monad 结构,其中后者允许用户自定义全局信息,举个例子,如果需要统计被匹配到的所有 identifier 数目,就需要定义 AlexUserState 包含一个累计 identifier 数目的条目。
	\paragraph{}
	而不声明 wrapper 时,用户只能使用最底层的 API。有趣的一点是,使用底层 API 时,用户需要声明更底层的函数输入类型和获取单个字符的函数。Alex 开放输入类型约定的主要原因在于,Haskell 标准库的 String 实现是非常低效的,在强调性能的场合,往往使用 Text 或 ByteString 这样的高性能字符串实现。在定义好最基本的函数之后,每个模式对应的 action 就应该有统一的类型 $AlexInput \to Int \to AlexReturn action$,或 $user \to AlexInput \to Int \to Maybe\ (AlexInput, Int, action)$,后者相比前者,增加了对于谓词的支持。
	\begin{minted}{haskell}
main = putStrLn $ show (1 :: Int)
	\end{minted}
	\printbibliography
\end{document}